\documentclass[a4paper,12pt]{article}

\usepackage[portuges]{babel}
\usepackage[latin1]{inputenc}
\usepackage{graphicx}

\begin{document}

\begin{center}
\Huge{Dilata{\c c}{\~ a}o}
\end{center}

\section{Introdu{\c c}{\~ a}o}

A medida que a temperatura sobe as medidas de um corpo aumentam gra{\c c}as {\`
a} maior agita{\c c}{\~ a}o de suas mol{\' e}culas. Este fen{\^ o}meno {\' e}
chamado de dilata{\c c}{\~ a}o t{\' e}rmica.

\section{Dilata{\c c}{\~ a}o Linear}

O aumento de temperatura dilata os corpos quantitativamente em fun{\c c}{\~ a}o
da equa{\c c}{\~ a}o (\ref{eq1}), da dilata{\c c}{\~ a}o quantitativa.

\begin{equation}
\Delta L=L\alpha\Delta T
\label{eq1}
\end{equation}

\section{Dilata{\c c}{\~ a}o Volum{\' e}trica}

Para os l{\' i}quidos onde n{\~ a}o podemos efetuar medi{\c c}{\~ o}es diretas,
usamos a equa{\c c}{\~ a}o volum{\' e}trica aplicando diretamente tr{\^ e}s
dimens{\~ o}es na equa{\c c}{\~ a}o.  Na equa{\c c}{\~ a}o (\ref{eq2})
$\beta = 3*\alpha$ j{\' a} que o componente de dilata{\c c}{\~ a}o linear
apareceria tr{\^ e}s vezes se fossemos usar a equa{\c c}{\~ a}o (\ref{eq1}).

\begin{equation}
\Delta V=V\beta\Delta T
\label{eq2}
\end{equation}

\section{Exemplo de dilata{\c c}{\~ a}o}

Um bast{\~ a}o de a{\c c}o ({\it coeficiente de dilata{\c c}{\~ a}o
$1,1*10^{-5}\  K^{-1}$}), com o aumento de temperatura,  dilata tamb{\' e}m seu
di{\^ a}metro como mostra a figura (\ref{figura}).
\begin{figure}[h]
\caption{Bast{\~ a}o de A{\c c}o}
\begin{picture}(380, 50)(0,0)
\put(20,45){\line(0,-1){30}}
\put(20,45){\line(1,0){120}}
\put(20,15){\line(1,0){120}}
\put(140,45){\line(0,-1){30}}
\put(140,45){\line(-1,-1){10}}
\put(135,45){\line(-1,-1){10}}
\put(130,45){\line(-1,-1){10}}
\put(125,45){\line(-1,-1){10}}
\put(120,45){\line(-1,-1){10}}
\put(115,45){\line(-1,-1){10}}
\put(110,45){\line(-1,-1){10}}
\put(105,45){\line(-1,-1){10}}
\put(100,45){\line(-1,-1){10}}
\put(95,45){\line(-1,-1){10}}
\put(90,45){\line(-1,-1){10}}
\put(85,45){\line(-1,-1){10}}
\put(80,45){\line(-1,-1){10}}
\put(75,45){\line(-1,-1){10}}
\put(70,45){\line(-1,-1){10}}
\put(65,45){\line(-1,-1){10}}
\put(60,45){\line(-1,-1){10}}
\put(55,45){\line(-1,-1){10}}
\put(50,45){\line(-1,-1){10}}
\put(45,45){\line(-1,-1){10}}
\put(40,45){\line(-1,-1){10}}
\put(35,45){\line(-1,-1){10}}
\put(20,15){\line(1,1){10}}
\put(25,15){\line(1,1){10}}
\put(30,15){\line(1,1){10}}
\put(35,15){\line(1,1){10}}
\put(40,15){\line(1,1){10}}
\put(45,15){\line(1,1){10}}
\put(50,15){\line(1,1){10}}
\put(55,15){\line(1,1){10}}
\put(60,15){\line(1,1){10}}
\put(65,15){\line(1,1){10}}
\put(70,15){\line(1,1){10}}
\put(75,15){\line(1,1){10}}
\put(80,15){\line(1,1){10}}
\put(85,15){\line(1,1){10}}
\put(90,15){\line(1,1){10}}
\put(95,15){\line(1,1){10}}
\put(100,15){\line(1,1){10}}
\put(105,15){\line(1,1){10}}
\put(110,15){\line(1,1){10}}
\put(115,15){\line(1,1){10}}
\put(120,15){\line(1,1){10}}
\put(125,15){\line(1,1){10}}
\put(200,50){\line(1,0){160}}
\put(200,50){\line(0,-1){40}}
\put(200,10){\line(1,0){160}}
\put(360,50){\line(0,-1){40}}
\put(360,50){\line(-1,-1){13}}
\put(353,50){\line(-1,-1){13}}
\put(346,50){\line(-1,-1){13}}
\put(339,50){\line(-1,-1){13}}
\put(332,50){\line(-1,-1){13}}
\put(325,50){\line(-1,-1){13}}
\put(318,50){\line(-1,-1){13}}
\put(311,50){\line(-1,-1){13}}
\put(304,50){\line(-1,-1){13}}
\put(297,50){\line(-1,-1){13}}
\put(290,50){\line(-1,-1){13}}
\put(283,50){\line(-1,-1){13}}
\put(276,50){\line(-1,-1){13}}
\put(269,50){\line(-1,-1){13}}
\put(262,50){\line(-1,-1){13}}
\put(255,50){\line(-1,-1){13}}
\put(248,50){\line(-1,-1){13}}
\put(241,50){\line(-1,-1){13}}
\put(234,50){\line(-1,-1){13}}
\put(227,50){\line(-1,-1){13}}
\put(220,50){\line(-1,-1){13}}
\put(213,50){\line(-1,-1){13}}
\put(346,10){\line(1,1){13}}
\put(339,10){\line(1,1){13}}
\put(332,10){\line(1,1){13}}
\put(325,10){\line(1,1){13}}
\put(318,10){\line(1,1){13}}
\put(311,10){\line(1,1){13}}
\put(304,10){\line(1,1){13}}
\put(297,10){\line(1,1){13}}
\put(290,10){\line(1,1){13}}
\put(283,10){\line(1,1){13}}
\put(276,10){\line(1,1){13}}
\put(269,10){\line(1,1){13}}
\put(262,10){\line(1,1){13}}
\put(255,10){\line(1,1){13}}
\put(248,10){\line(1,1){13}}
\put(241,10){\line(1,1){13}}
\put(234,10){\line(1,1){13}}
\put(227,10){\line(1,1){13}}
\put(220,10){\line(1,1){13}}
\put(213,10){\line(1,1){13}}
\put(206,10){\line(1,1){13}}
\put(200,10){\line(1,1){13}}
\put(15,5){Bast{\~ a}o antes da dilata{\c c}{\~ a}o}
\put(220,0){Bast{\~ a}o ap{\' o}s a dilata{\c c}{\~ a}o}
\label{figura}
\end{picture}
\end{figure}
O aumento de temperatura {\' e} representado na tabela (\ref{tabela}), junto
com o aumento linear e volum{\' e}trico do bast{\~ a}o. Digamos que o
bast{\~ a}o tem 1 m de comprimento e 25 cm de di{\^ a}metro da base, a 0 graus
Celsius.

\newpage

\begin{table}[ht]
\center
\caption{Medidas em rela{\c c}{\~ a}o {\` a} Temperatura}
\begin{tabular}{|c|c|c|}
\hline
Temperatura&Comprimento do Bast{\~ a}o&Volume do Bast{\~ a}o\\
($^oC$)&(m)&($m^{3}$)\\
\hline
 0   & 1.0000 & 0.392699 \\
\hline
 100 & 1.0011 & 0.393995 \\
\hline
 200 & 1.0022 & 0.395291 \\
\hline
 300 & 1.0033 & 0.396587 \\
\hline
 400 & 1.0044 & 0.397883 \\
\hline
 500 & 1.0055 & 0.399179 \\
\hline
 600 & 1.0066 & 0.400475 \\
\hline
\end{tabular}
\label{tabela}
\end{table}
Nos gr{\' a}ficos (figura (\ref{grafico})) desenvolvidos a partir da
tabela (\ref{tabela}) podemos verificar a igualdade de $\beta =3*\alpha$.
\begin{figure}[h]
\includegraphics[scale=0.55]{grafico1.eps}
\includegraphics[scale=0.55]{grafico2.eps}
\caption{Exemplo de Dilata{\c c}{\~ a}o}
\label{grafico}
\end {figure}

Para o desenvolvimento deste trabalho usou se como refer{\^ e}ncia o livro
\cite{X}, e o website \cite{Y}.

\begin{thebibliography}{1}

\bibitem{X} D. Halliday, R. Resnick, J. Walker; Fundamentos de F{\' i}sica.

\bibitem{Y} Funda{\c c}{\~ a}o CAPES; http://www.capes.gov.br.

\end{thebibliography}

\end{document}

