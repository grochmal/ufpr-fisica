\documentclass[12pt]{article}

\usepackage[portuges]{babel}
\usepackage[latin1]{inputenc}

\begin{document}
\begin{center}
\large{O PAPEL DOS ALQUILOGLICEROIS, SQUALENE E ?CIDOS INSATURADOS DE OMEGA-3
NO TRATAMENTO DE INFEC??ES DAS VIAS A?REAS SUPERIORES - REGULA??O DOS
MECANISMOS DO SISTEMA IMUNOL?GICO INATO.}
\end{center}

 A inflama??o ? uma complexa rea??o do organismo causada por fatores
patogenicos externos (p. ex. Bacter?as), les?es teciduais ou dist?rbios
do pr?prio sistema imunol?gico [1]. Nos ?timos anos o desenvolvimento de
pesquisas do sistema imunol?gico trouxe mais informa??es sobre o papel
e os mecanismos da resposta imunol?gica. Al?m disso, abriu caminhos
para o uso das subst?ncias naturais, com propriedades imunoreguladoras, no
tratamento de var?as doen?as. A maior vantagem desse tipo de terapia ?
a falta de efeitos colaterais e a estimula??o dos mecanismos endogenos
da resposta imunol?gica, especialmente, do sistema imunol?gico inato
[2, 3]. As propriedades imunoreguladoras das subst?ncias, que contem
?cidos insaturados de Omega-3, squalene e alquiloglicerois foram
confirmados atrav?s de var?os estudos cl?nicos, pesquisas com animais, testes
c?lulares in vitro e observa??es  epidemiol?gicas [3, 4, 5, 6].
 As emuls?es lip?dicas, que contem ?cidos insaturados de Omega-3, cada
vez com mais frequ?ncia s?o usados na alimenta??o parenteral [7, 8, 9,
10, 11, 12]. Pesquisas com animais monstraram a maior efic?cia das
administra??es intravenosas de ?cidos insaturados Omega-3 e
alquiloglicerois no tratamento de septicemia causada pelo E. coli, em compara??o com
as emuls?es lip?dicas parenterais de Omega-6 [11, 12]. Os ?cidos
insaturados de Omega-3 e alquiloglicerois causaram mais rapida elimina??o das
bacter?as E. coli, atrav?s do aumento das propriedades fagocit?rias dos
neutr?filos e dos macr?fagos dos pulm?es (c?lulas de Kupfer) e, al?m
disso, diminuiram a intensidade do processo inflamat?rio. Os resultados
das pesquisas sugerem a possibilidade do uso de ?leo de figado de peixes
mar?nios no combate das infec??es bacter?nas.
 Um dos poss?veis mecanismos da atividade antiinflamat?ria dos ?cidos
insaturados de Omega-3 ? a inibi??o das enzimas ciclooxigenase e
lipooxigenase. Essa inibi??o causa a diminui??o da s?ntese de derivados do
?cido aracidico, como: prostaglandina (PGE2), tromboxane (TXA2),
leucotriene (LT4), lipoxina e, ao mesmo tempo, o aumento da libera??o das
prostaglandinas, que inibem o processo inflamat?rio, como: prostaglandina
(PGE3) e leucotriene (LT5) [13, 14, 15]. Outros pesquisadores sugerem, que
esse efeito antiinflamat?rio ? potencializado pela diminui??o da
s?ntese das citoc?ninas inflamat?rias (IL-2, IL-6, IL-8, IL-12, TNF-alfa) com
pequena influ?ncia para a citoc?nina antiinflamat?ria (IL-10) [14, 15,
16, 17, 18, 19, 20].
 As observa??es epidemiol?gicas, os estudos cl?nicos e os experimentos
nas c?lulas in vitro aprovaram uso dos ?leos obtidos de peixes mar?nios
como imunoreguladores da resposta imunol?gica inata [2].
 No presente trabalho foi avaliada a efic?cia do Biomarine 570, que
contem o ol?o de figado de tubar?o (conte?do: 120mg de squalene, 120mg de
alquiloglicerol , 25mg de ?cidos graxos de Omega-3, 50 UI de vit. A, 5
UI de vit. D), no tratamento das recorrentes infec??es das vias
aereassuperiores.  Foram avaliados os sintomas cl?nicos dos pacientes e algums
par?metros do sistema imunol?gico, como: atividade dos neutr?filos,
algums par?metros do sistema de complemento e a percentagem das
subpopula??es dos limf?citos T (CD3, CD4, CD8), dos limf?citos B (CD19) e das
c?lulas NK (CD16/CD56).

\begin{center}
\large{MATERIAL E METODOS}
\end{center}

 Da pesquisa participaram 19 pessoas: 11 mulheres e 8 homens com idades
entre 19 e 55 anos (miedia de 38,4 +- 11,76 anos). A sele??o das
pessoas para os grupos de pesquisa foi feita com base em entrevista e exame
cl?nico. Assim foram escolhidos os pacientes que apresentavam
recorrentes infec??es de vias aereas superiores, sem nenhuma doen?a sist?mica e
sem uso de nenhum medicamento. O medicamento foi administrado em dose de
3 c?psulas 3 vezes por dia durante 2 meses. Foi efetuada coleta de 2 ml
de sangue de cada dos pacientes antes e ap?s o tratamento, para
realiza??o dos exames imunol?gicos e avalia??o morfol?gica do sangue
perif?rico. Ap?s o final do tratamento os pacientes foram observados e avaliados
clinicamente durante 6 meses.
 A atividade dos neutr?filos, que  foi avaliada atrav?s da sua
capacidade de produ??o das formas reativas de oxigenio (RFT): sem estimulo, com
estimuladores receptoro- dependentes
(formolo-metionilo-leucilo-fenilalanina - fMLP e zimosan opsonisado - OZ) e com estimuladores
receptoro-independentes (?ster de forbol - PMA) pelo m?todo de quimioluminesc?ncia
do sangue total [21]. Este no aparelho MLX Microtiter Plate
Luminometer. Os valores da quimioluminesc?ncia do sangue total foram corigidos
pelo n?mero total dos neutr?filos e pela concentra??o da hemoglobina.
Foram expressos pelas unidades convencionais da quimioluminesc?ncia RLUmax
(Relative Light Units Max) de acordo com formula:

 CL calculada = CL medida[RLUmax] x Hb [%] / (WBC [10^3/mi*l.] x PMN
[%])

 Onde:
 WBC -   n?mero total das celulas brancas (10^3/mi*l)
 CL     -   quimioluminesc?ncia (RLUmax)
 Hb     -   hemoglobina (%)
 PMN -   neutr?filos do sangue total (%)

 Os valores de C3c e de C4 do sistema de complemento foram avaliados
pelo m?todo de nefelometria com o uso dos reagentes de Behring, e a
atividade da via cl?ssica do complemento (CH50) pelo m?todo, modificado, de
Mayer.
 A an?lise da percentagem das subpopulac?es dos limf?citos T (CD3, CD4,
CD8), dos limf?citos B (CD19) e das c?lulas NK (CD16/CD56) no sangue
perif?rico, foi feita no fluxocit?metro FACS Calibur Becton-Dickinson e
foram usados marcados anticorpos monoclonais Becton-Dickinson.
 Para todos os par?metros foi calculada m?dia aritm?tica e o desvio
padr?o (SD). A alia??o estat?stica foi feita de acordo com as seguintes
testes: Kolmogorow-Smirnow com corre??o de Lilliefors para an?lise de
concord?ncia dos resultados obtidos com os valores de refer?ncia;
Fisher-Snedecor, t-Student, Cohran-Cox e t-Student para var?aveis, com objetivo
de an?lise dos resultados esperados entre os grupos estudados. Para
resultados estatisticamente aceitos foi adotado p <= 0,05. Os pacientes
foram divididos em grupos pela idade para avaliar a normaliza??o, ou n?o,
dos analisados par?metros imunol?gicos em compara??o com valores de
refer?ncia para cada m?todo.



 RESULTADOS

 Avalia??o cl?nica

 No final da terapia com Biomarine 570 foi observada a diminui??o da
frequ?ncia das infec??es das vias aereas superiores em compara??o com a
frequ?ncia antes de tratamento. A m?dia da frequ?ncia das infec??es caiu
de 0,65/m?s antes de tratamento para 0,52/m?s durante 6 meses de
observa??o ap?s a terapia. Em 7 dos 19 pacientes n?o apresentara se nenhuma
recorr?ncia de infec??o das vias aereas superiores durante 6 meses de
observa??o.
 Durante a terapia com o medicamento n?o foram observados efeitos
colaterais.

 Exames imunol?gicos

 A an?lise da percentagem das subpopula??es dos limf?citos do sangue
perif?rico monstrou a baixa percentagem dos limf?citos B (CD19+) e dos
limf?citos T CD8+ nas pessoas doentes. Ap?s o tratamento foi constatado o
aumento da percentagem dos limf?citos T CD8+ e dos limf?citos B (CD19)
at? valores normais, observados nas pessoas saud?veis. A percentagem de
cada tipo das subpopula??es dos limf?citos T, dos limf?citos B das
c?lulas NK e da rela??o CD4/CD8 est?o na Tabela 1.
 Antes do tratamento foi observada baixa produ??o de RFT pelos
neutr?filos com estimula??o OZ (zimosan opsonizado) e com estimula??o PMA
(?ster de forbol - um estimulador receptoro-independente, que estimula os
neutr?filos pela oxidase de NADPH). Ap?s o tratamento a produ??o de RFT
pelos neutr?filos com estimula??o OZ e PMA aumentou significativamente -
os valores de quimioluminesc?ncia aproximaram-se aos valores de
refer?ncia. Os valores de quimioluminesc?ncia  para os neutr?filos n?o
estimulados e estimulados com fMLP tamb?m aumentaram significativamente em
compara??o com os valores observados antes da terapia. A an?lise das
m?dias dos valores de RFT dos neutr?filos n?o estimulados e com estimula??o
fMLP, OZ e PMA da fase ativa da doen?a (antes de tratamento) e da fase
da remiss?o da doen?a (ap?s de 2 meses de tratamento), junto com os
valores das pessoas saud?veis monstra a Figura 1.
 Os nive?s do C3c e C4 e a atividade hemol?tica (CH50) do sistema de
complemento foram aumentados no grupo dos pacientes na fase ativa da
doen?a, em compara??o com os valores das pessoas sad?as. Ap?s o tratamento
os valores do C4 e CH50 abaixaram-se significativamente em compara??o
com os nive?s da fase ativa da doen?a, mas n?o voltaram para os nive?s
normais (Figura 2).

 DISCUSS?O

 Os medicamentos, que contem alquiloglicerois, squalene, ?cidos
insaturados de Omega-3 e vit. A e D s?o extraidos do ?leo de figado de
tubar?o. Os ?cidos insaturados de Omega-3 n?o s?o sintetizados de novo no
organismo humano. O substrato dos ?cidos insaturados de Omega-3 ? o ?cido
alfa-linole?co, que, sujeito aos processos enzim?ticosda, da origem a
dois produtos: o ?cido eicosapentanoico (EPA) e o ?cido docosahexanoico
(DHA). Uma dieta rica em Omega-3 causa a diminui??o dos nive?s do ?cido
arac?dico e , ao mesmo tempo, o aumento dos nive?s dos ?cidos
eicopentanoico e docosahexanoico. Assim diminui a produ??o dos derivados do
?cido arac?dico (PGE2 e LT4) e aumenta a s?ntese dos metab?litos do ?cido
eicopentanoico (PGE3 e LTA5), que tem menor atividade biol?gica de que
seus an?logos. Al?m disso, EPA e DHA competitivo inibem a conver??o do
?cido arac?dico em prostaglandinas [13, 14]. Deste modo, uma dieta rica
em Omega-3 restringe, naturalmente, processos inflam?torios cr?nicos
[14].
 As nossas pesquisas monstraram que a administra??o do medicamento
causa a diminui??o da frequ?ncia de infec??es das vias aereas superiores e
a normaliza??o de algums par?metros imunol?gicos analisados. A
capacidade dos neutr?filos para produ??o das formas reativas de oxigenio (RFT)
foi usada para a avalia??o da atividade dos neutr?filos, esta an?lise
monstrou que a atividade dos neutr?filos estimulados com OZ e PMA
aumentou ap?s o tratamento.  As nossas observa??es foram, tamb?m, confirmados
pelas pesquisas nos animais com septicemia causada pelo E. coli do
prof. Lanz-Jacob e col. [11].  A melhor resposta dos neutr?filos a
estimula??o com OZ pode ser uma consequ?ncia da melhor e mais efic?z
opsoniza??o das bact?rias e do aumento da capacidade e da efic?cia da oxidase
NADPH (enzima respons?vel pela produ??o de RFT nos neutr?filos ativados).
Essa hip?tese ? sustentada tamb?m pelo fato de que os neutr?filos com
estimula??o PMA aumentaram sua atividade e produziram mais RFT
(estimula??o direta da oxidase NADHP), como tamb?m, pelas pesquisas de Robinson
e Ferrante [22, 23]. Ent?o n?s vem a pergunta, sera que o aumento da
produ??o de RFT nos pacientes em tratamento com Biomarine 570 pode
perturbar os mecanismos antioxidativas naturais. Presentes nos compartimentos
intra e extracelulares os ?cidos insaturados e squalene s?o subst?ncias
com propriedades antioxidativas. Os ?cidos insaturados e squalene s?o,
de um lado, os substratos para a repara??o da membrana celular j?
danificada e, de outro lado, s?o ''um escudo'' contra os RFTs extracelulares
protegendo os lip?dios da membrana da peroxida??o [2]. Os componentes
do ?leo de figado do tubar?o s?o elementos naturais da membrana celular
e protegem a c?lula de radicais livres, servindo como substratos para a
repara??o da membrana. De fato, os ?cidos insaturados e squalene s?o,
tamb?m, os componentes naturais da membrana lipidica dos mitoc?ndrios, a
fonte principal da gera??o dos RFTs (cadeia respirat?ria), o que
aumenta a signific?ncia deles como antioxidantes [2].
 Os componentes do sistema de complemento C3c e C4 s?o sens?veis
marcadores do processo inflamat?rio (C4 - ativa??o da via cl?ssica do sistema
de complemento, C3c - ativa??o da via cl?ssica e alternativa). A
an?lise dos nive?s dos componentes do sistema de complemento (diminui??o dos
nive?s de C4) e da atividade hemol?tica (diminui??o do valor CH50)
aponta na remiss?o do processo inflamat?rio. As nossas pesquisas anteriores
nos pacientes com Recorrentes Aftas Odontol?gicas confirmaram, tamb?m,
que os ?cidos insaturados, squalene e alquiloglicerois restringem o
processo inflamat?rio. Os resultados foram parecidos com os resultados dos
pacientes com as Recorrentes Infec??es das Vias Respirat?rias
Superiores e monstraram a diminui??o dos nive?s dos componentes do sistema de
complemento C4 e da atividade hemol?tica CH50, como tambem, a diminui??o
da percentagem da subpopula??o dos limf?citos T com fen?tipo HLA-DR. A
diminui??o da percentagem dos limf?citos ativados T CD3/HLA-DR+ aponta,
tamb?m, para a remiss?o da doen?a. Foi observado, tamb?m, o aumento da
resposta dos neutr?filos para fMLP ap?s a ativa??o de TNF-alfa, que
indiretamente aponta para a remiss?o dos fatores inflamat?rios, que causam
a ativa??o dos neutr?filos in vivo. Os resultados obtidos confirmam as
propriedades antiinflamat?rias do Biomarine 570 [2, 24].
 Resumindo, uma dieta rica em ?cidos insaturados de Omega-3, squalene e
alquiloglicarois pode auxiliar o sistema imunol?gico do organismo na
luta contra as infec??es bacter?anas e na restri??o dos processos
inflamat?rios cr?nicos.
\end{document}

