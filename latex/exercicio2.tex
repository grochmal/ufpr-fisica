\documentclass{article}

\usepackage[portuges]{babel}
\usepackage[latin1]{inputenc}
\usepackage{graphicx}

\begin{document}

\begin{figure}[h]
\includegraphics{exercicio2.eps}
\caption{Posi{\c c}{\~ a}o ${\it x}$, velocidade ${\it v}$ e
acelera{\c c}{\~ a}o ${\it a}$ vs tempo ${\it t}$}
\label{fig1}
\end{figure}

Quando a acelera{\c c}{\~ a}o de um corpo {\' e} constante, ou seja
\begin{equation}
{\it a}=cte
\label{eq1}
\end{equation}
ent{\~ a}o a velocidade {\it v} depende linearmente de tempo
\begin{equation}
v=v_0+at
\label{eq2}
\end{equation}
e a posi{\c c}{\~ a}o {\it x} {\' e} a fun{\c c}{\~ a}o quadratica
\begin{equation}
x=x_0+vt+\frac{1}{2}at^2
\label{eq3}
\end{equation}
onde ${\it v_0}$ {\' e} a velocidade inicial. Os gr{\' a}ficos de {\it x(t)},
{\it v(t)} e {\it a(t)} em caso {\it a}=1m/s s{\~ a}o dados na figura
(\ref{fig1}).
\end{document}

