\documentclass[a4paper,12pt]{article}

\usepackage[portuges]{babel}
\usepackage[latin1]{inputenc}

\begin{document}

\begin{figure}[h]
\begin{picture}(250, 36)(0,0)
\put(125,24){${\bf F}_{21}$}
\put(225,24){${\bf F}_{12}$}
\put(150,18){\circle{12}}
\put(210,18){\circle{12}}
\put(146,15){+}
\put(206,15){+}
\put(144,18){\vector(-1,0){20}}
\put(216,18){\vector(1,0){20}}
\put(140,0){$q_{1}$}
\put(210,0){$q_{2}$}
\label{figura1}
\end{picture}
\caption{Cargas de mesmo sinal se repelem}
\end{figure}

\begin{figure}[h]
\begin{picture}(250, 36)(0,0)
\put(155,24){${\bf F}_{21}$}
\put(190,24){${\bf F}_{12}$}
\put(150,18){\circle{12}}
\put(210,18){\circle{12}}
\put(146,15){+}
\put(206,15){--}
\put(156,18){\vector(1,0){20}}
\put(204,18){\vector(-1,0){20}}
\put(140,0){$q_{1}$}
\put(210,0){$q_{2}$}
\label{figura2}
\end{picture}
\caption{Cargas de sinais opostos se atraem}
\end{figure}

A For{\c c}a ${\bf F_{12}}$ exercida por uma carga ${\bf q_1}$ sobre uma outra
carga ${\bf q_2}$ {\' e} dada como
\begin{equation}
F_{12}=k\frac{q_1q_2}{r_{12}^2}\hat r_{12}
\label{eq1}
\end{equation}
onde ${\bf \hat r}_{12}$ {\' e} o vetor unit{\' a}rio que aponta de ${\it q_1}$
para ${\it q_2}$, {\it k} {\' e} a constante Coulumb relacionada com a
permissividade do v{\' a}cuo
\begin{equation}
k=\frac{1}{4\pi \epsilon_0}
\label{eq2}
\end{equation}
A dire{\c c}{\~ a}o das for{\c c}as {\' e} mostrada nas figuras 1 e 2

O campo el{\' e}tirco num ponto {\it P} na vizinhan{\c c}a de uma carga
puntiforme {\bf q} que est{\' a} na origem de sistemade coordenadas {\' e} dado
como
\begin{equation}
E=k\frac{q}{r^2}\hat{r}
\label{eq3}
\end{equation}
onde ${\bf \hat r}$ {\' e} o vetor unit{\' a}rio que aponta da carga para o
ponto {\it P}.

O campo el{\' e}tirco {\bf E} {\' e} o negativo do gradiente do potencial
{\it V}
\begin{equation}
E=-\nabla V=\left(\frac{\partial V}{\partial x}i+\frac{\partial V}{\partial
    y}j+\frac{\partial V}{\partial z}k\right).
\label{eq4}
\end{equation}

\end{document}

