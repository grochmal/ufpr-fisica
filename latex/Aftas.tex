\documentclass[12pt]{article}

\usepackage[portuges]{babel}
\usepackage[latin1]{inputenc}

\begin{document}
\begin{center}
\large{A AVALIA??O DA EFIC?CIA DO ?LEO DE FIGADO DE TUBAR?O NO TRATAMENTO DAS
RECORRENTES AFTAS ODONTOL?GICAS (RAS), NOS ASPECTOS DE PESQUISAS CL?NICAS E
IMUNOL?GICAS.}
\end{center}

 As Recorrentes Aftas Odontol?gicas s?o as mais frequentes doen?as da mucosa da
boca. Uma das caracter?sticas da doen?a ? o peri?dico aparecimento e
desenvolvimento de dolorosas feridas na mucosa da boca, com caracter?sticos
per?odos de remiss?o da doen?a. Estes per?odos podem durar de algums dias at?
algums meses.

 At? agora a etiologia das Recorrentes Aftas Odontologicas n?o ? conhecida
exatamente. Var?os fatores locais e gerais podem causar aparecimento,
desenvolvimento e peri?dos de recorr?ncia da doen?a, por exemplo: trauma,
conte?do da saliva, stress, fatores humorais, fatores gen?ticos, dist?rbios
imunologicos, infec??es virais e bacterianas. Al?m disso, a coexist?ncia de
algumas doen?as sistemicas pode ter rela??o com a patog?nese das Aftas
recorrentes [10].

 Algums autores acham, que a disfun??o do sistema imunol?gico pode exercer
papel principal no desenvolvimento dessa doen?a [10].  Infelizmente, apesar de
vari?s pesquisas do sistema imunol?gico e do papel dele nas Recorrentes Aftas
Odontol?gicas,  n?o foi poss?vel esclarecer, exatamente, o patomecanismo desta
doen?a. Como tamb?m n?o foi poss?vel estabelecer procedimentos exatos de
tratamento, que poderiam trazer cura ou, pelo menos, minimizar os sintomas e a
frequencia de recorr?ncia das aftas. Medicamentos locais s? minimizam s?ntomas
e os medicamentos sist?micos trazem s? remiss?es da doen?a, que duram apenas
algums meses  [6, 11, 12, 13, 14, 15, 21, 22, 27, 28].  Nenhum medicamento
previne a recor?ncias da doen?a a longo prazo. At? este momento foram usados,
no tratamento sistemico, tanto imunosupressores como imunoestimuladores [9, 14,
22, 23, 24, 27, 28]. Os primeiros, apesar da sua, certa, efic?cia, n?o s?o
indicados para tratamento a longo prazo por causa de seus efeitos
colaterais. Os imunoestimuladores s?o menos eficazes, mas podem ser usados no
tratamento coadjuvante por causa da sua positiva influ?ncia no organismo como
um todo.  As  pesquisas de um medicamento, pouco t?xico e ao mesmo tempo t?o
eficaz que, se usado periodicamente, poderia prevenir as recorr?ncias e
suavizar os sintomas da doen?a, s?o realizadas at? presente momento.

 Essas propriedades parece ter Biomarine 570, que contem ol?o de figado de
tubar?o (conte?do: 120mg de squalene, 120mg de alquiloglicerois, 25mg de ?cidos
graxos Omega-3, 50 UI de vit. A, 5 UI de vit. D), que tem a??o reguladora no
sistema imunol?gico. Os componentes naturais do ol?o de figado de tubar?o
melhoram a resposta imunol?gica inata do organismo por diferentes
mecanismos. Squalene estimula a resposta imunol?gica celular e humoral, protege
as celulas contra a a??o destrutiva das formas reativas do oxigenio. Al?m
disso, exerce direta atividade antibacteriana (inibe crescimento bacteriano),
como tamb?m, a??o indireta por "opsoniza??o" das celulas bacterianas,
facilitando a fagocitose [1, 2, 16].  Os alquiloglicerois regulam a atividade
das celulas do sistema imunol?gico: estimulam a hematopoese, a fagocitose e a
produ??o dos anticorpos. Demonstram, tamb?m, direta atividade antibacter?ana e
antifungic?da [31]. Uma dieta que contem os polinsaturados ?cidos graxos de
Omega-3 inibe a s?ntese das IL-1, IL-2, IL-6, $TNF-\alpha$, $INF-\gamma$
atrav?s dos limf?citos do sangue perif?rico e a redu??o da s?ntese de PGE-2
[5, 7, 30]. Foi constatada tamb?m a diminui??o da express?o do HLA-DP, HLA-DQ e
HLA-DR e ainda do ICAM-1 e do LFA-1 nos mon?citos [4].

 No presente trabalho foi avaliada a efic?cia do Biomarine 570, que contem o
ol?o de figado de tubar?o, no tratamento das Recorrentes Aftas Odontol?gicas,
pela observa??o dos sintomas cl?nicos e a avalia??o de algums par?metros de
resposta imunol?gica nos pacientes, que sujeitaram-se ao tratamento durante 3
meses.

\begin{center}
\large{MATERIAL E METODOS}
\end{center}

 A pesquisa foi realizada com a autoriza??o da Comiss?o de ?tica de Pesquisas
Scient?ficas do Departamento de M?dicina da Univercidade da cidade de Lodz. Da
pesquisa participaram 25 pessoas: 16 mulheres e 9 homens com idades entre 19 e
60 anos (miedia de 39,4 $\pm$ 13,76 anos). A sele??o das pessoas para os grupos
de pesquisa foi feita com base em entrevista e exame cl?nico. Assim foram
escolhidos os pacientes que apresentavam a recorr?ncia das aftas pelo menos uma
vez por mes, sem nenhuma doen?a sist?mica e sem uso de nenhum medicamento.

 Os pacientes foram observados durante 2 meses, antes de come?arem com o
tratamento (e a pesquisa), para calcular a m?dia, da quantidade, das aftas e a
frequencia das recorr?ncias por mes. Foi efetuada coleta de 2 ml de sangue dos
pacientes: plasma (tubo com heparina) e soro (tubo com gel), antes e ap?s o
tratamento, para realiza??o dos exames imunol?gicos e avalia??o morfol?gica do
sangue perif?rico. Cada paciente estava na fase aguda da doen?a (2-4 dias a
partir do aparecimento das aftas) no momento do come?o do tratamento e da
pesquisa.

 Foram avaliados os seguintes par?metros imunol?gicos:

 * A Atividade dos neutr?filos, que  foi avaliada atrav?s da sua capacidade de
produ??o das formas reativas de oxigenio (RFT): sem estimulo, com estimuladores
receptoro- dependentes (formolo-metionilo-leucilo-fenilalanina - fMLP e zimosan
opsonisado - OZ) e com estimuladores receptoro-independentes (?ster de forbol -
PMA) pelo m?todo de quimioluminesc?ncia do sangue total [18]. Este no aparelho
MLX Microtiter Plate Luminometer. Os valores da quimioluminesc?ncia do sangue
total foram corrigidos pelo n?mero total dos neutr?filos e pela concentra??o da
hemoglobina. Foram expressos pelas unidades convencionais da
quimioluminesc?ncia $RLU_{max}$ (Relative Light Units Max) de acordo com formula:

\begin{equation}
CL_{calculada} = CL_{medida}[RLU_{max}]*\frac{Hb[\%]}{WBC[10^3/\mu l.]*PMN[\%]}
\end{equation}

Onde:

WBC -   n?mero total das celulas brancas ($10^3/\mu l$)

CL  -   quimioluminesc?ncia ($RLU_{max}$)

Hb  -   hemoglobina (\%)

PMN -   neutr?filos do sangue total (\%)

 * A an?lise da percentagem das subpopulac?es dos limf?citos T (CD3, CD4, CD8),
dos limf?citos B (CD19), dos limf?citos NK (CD16/CD56) e dos limf?citos T CD3
com express?o HLA-DR no sangue perif?rico, foi feita no fluxocit?metro FACS
Calibur Becton-Dickinson e foram usados anticorpos monoclonais marcados
Becton-Dickinson.

 * Os valores C3c e C4 do sistema de complemento foram avaliados pelo m?todo de
nefelometria com o uso dos reagentes de Behring, e a atividade da via cl?ssica
do complemento (CH50) pelo m?todo, modificado, de Mayer.

 O grupo de controle para a quimioluminesc?ncia dos neutr?filos e para o
sistema de complemento foi constituido de 19 pessoas saud?veis, que nunca
sofreram das Recorrentes Aftas Odontol?gicas. A percentagem das subpopulac?es
dos limf?citos das pessoas doentes foram comparados com os valores de
refer?ncia, para a popula??o polonesa [29].

 Os pacientes foram instru?dos sobre modo de administra??o do medicamento
(c?psulas derretidas na boca at? estourar, ent?o o ol?o devia ser espalhado
pela mucosa da boca e, s? depois, engolido). O medicamento foi administrado: 3
c?psulas, 3 vezes por dia durante 3 meses.  A cada 2 semanas foi feita uma
avalia??o cl?nica das aftas (quantidade, tamanho, dor, cicatriza??o) e
impress?es subjetivas dos pacientes, estas avaliadas segundo a  escala: 0 -
falta de melhora, 1 - melhora. Ap?s o tratamento os pacientes foram avaliados
uma vez por mes, durante dois meses.

Para todos os par?metros foi calculada a m?dia aritm?tica e o desvio padr?o
(SD) e para a avialia??o cl?nica o erro padr?o da m?dia (SEM). A avalia??o
estat?stica foi feita de acordo com os seguintes testes: Kolmogorow-Smirnow com
corre??o de Lilliefors, Fisher-Snedecor,  t-Student, Cohran-Cox e t-Student
para var?aveis. Para resultados estatisticamente aceitos foi adotado p <= 0,05.

\begin{center}
\large{RESULTADOS}
\end{center}

 AVALIA??O CL?NICA.  No final do tratamento 3 pacientes do grupo de 25 pessoas
ainda tinham aftas, os restantes n?o apresentavam nenhuma altera??o na mucosa
da boca. Foi observada a diminui??o da frequ?ncia de aparecimento das aftas no
primeiro e no terceiro mes de tratamento em compara??o com a situa??o antes do
tratamento.  A frequ?ncia de aparecimentodas aftas diminuiu de 1,56/mes antes
do tratamento para 0,95/mes no primeiro mes ap?s o tratamento (Tab.1).  A
quantidade das aftas diminuiu durante o tratamento, sendo mais baixa no
primeiro mes ap?s o tratamento. A avalia?ao subjetiva dos pacientes sobre a
efic?cia do tratamento situava-se entre 56-67\% durante o tratamento, mas
abaixou para 33\% no primeiro m?s ap?s o tratamento, o que n?o esta de acordo
com  a redu??o da frequ?ncia e da quantidade das aftas. No peri?do de 2 meses
ap?s o tratamento foi observado, que em 4 pacientes n?o aconteceu a recorr?ncia
das aftas. 3 pacientes n?o responderam positivamente ao tratamento, nestes n?o
foi obsevada nenhuma melhora cl?nica. Nos restantes, do grupo, foi observado o
aumento da frequ?ncia e da quantidade das aftas no segundo m?s ap?s o
tratamento, mas com menor intensidade.

 EXAMES IMUNOL?GICOS. A an?lise da percentagem das subpopula??es dos limf?citos
do sangue perif?rico, dos doentes,  monstrou uma diminui??o dos nive?s de
limf?citos B e aumento dos limf?citos T com marcadores HLA-DR+, caracter?sticos
para limf?citos estimulados,  nas pessoas doentes. Ap?s o tratamento foi
constatado aumento dos limf?citos T CD3 e limf?citos B, e diminui??o dos nive?s
de limf?citos T HLA-DR+ em compara??o com os valores anteriores ao tratamento
(Tab.2).

 Nos exames antes do tratamento foi observado aumento na produ??o do RFT pelos
neutr?filos n?o estimulados e estimulados com estimuladores fMLP, mas os
neutr?filos estimulados com estimuladores PMA geraram baixa quantidade de RFT
em compara??o com as pessoas saud?veis.  Ap?s o tratamento a produ??o do RFT
pelos neutr?filos estimulados com estimuladores OZ aumentou significativamente
e os valores de quimioluminesc?ncia (RFT) dos neutr?filos estimulados com
estimuladores PMA estavam dentro dos valores normais. Para neutr?filos n?o
estimulados e estimulados com estimuladores fMLP os valores de
quimioluminesc?ncia (RFT) n?o mudaram em compara??o com os valores de RFT
obtidos nos pacientes antes do tratamento (Tab.3).

 Os nive?s do C3c e do C4 e da atividade hemol?tica (CH50) do sistema de
complemento estavam aumentados no grupo dos pacientes na fase aguda da doen?a,
em compara??o com as pessoas sad?as. Ap?s o tratamento os valores do C4 e do
CH50 abaixaram-se significativamente em compara??o com os nive?s da fase aguda,
mas n?o voltaram para nive?s normais (Fig. 1).

\begin{center}
\large{DISCUSS?O}
\end{center}

 Da pesquisa participaram pessoas que sofreriam da forma aguda das Recorrentes
Aftas Odontol?gicas. Durante os primeiros 2 meses de tratamento n?o houve
redu??o significativa na frequ?ncia e quantidade das aftas, mas foi observada a
maior melhora subjetiva, declarada pelos pacientes pacientes (56-67\%).  No
primeiro m?s ap?s o tratamento na avalia??o cl?nica foi observada a menor
frequ?ncia e quantidade das aftas, mas o n?mero dos pacientes, que declarava
melhora, caiu para 33\%. A discrep?ncia entre a avalia??o cl?nica e subjetiva
dos pacientes pode ser causada pela espectativa em rela??o ? efic?cia do
medicamento, no come?o do tratamento, e uma certa desilus?o com o efeito final
(Apesar dos pacientes estarem informados sobre a falta das informa??es sobre a
efic?cia do medicamento, a maioria esperava a cura total.).

 As Recorrentes Aftas Odontol?gicas s?o uma doen?a com etiologia complexa, na
qual os dist?rbios imunol?gicos tem papel importante. Foram avaliados os
seguintes par?metros do sistema imunol?gico: os nive?s das subpopula??es dos
limf?citos T, limf?citos B, celulas NK e limf?citos T com express?o HLA-DR,
gera??o de RFT pelos neutr?filos, os nive?s do C3c e do C4, e a atividade
hemol?tica (CH50) do sistema de complemento. Foi aplicado um plano de
pesquisas, que permitiu a avalia??o pr?via do sistema imunol?gico como
patomecanismo de desenvolvimento das aftas. A avalia??o dos par?metros
imunol?gicos nos pacientes ap?s o tratamento e estudos comparativos com os
valores obtidos antes do tratamento podem apontar que existe a influ?cia do
medicamento nos par?metros pesquisados e que ele apresenta uma certa efic?cia,
causando a remiss?o das aftas.

 Os neutr?filos do sangue perif?rico dos pacientes tanto antes como depois do
tratamento estavam estimulados e apresentavam aumentada produ??o do RFT pelas
c?lulas sem estimulo e estimuladas com fMLP. Ap?s o tratamento melhorou a
resposta dos neutr?filos estimulados com OZ e PMA. A obten??o dos valores de
RFT do sangue total exluiu a possibilidade de estimulo dos neutr?filos pelo
propr?o processo de isolamento das c?lulas.  A exagerada atividade dos
neutr?filos no sangue perif?rico pode ser causada pelo TNF-$\alpha$ end?geno,
do qual nive?s aumentados foram observados nos doentes com aftas recorrentes
[3, 19, 25]. A excessiva produ??o do RFT in vivo pode ter var?as consequ?ncias:
destrui??o das c?lulas do propr?o organismo, altera??o das propriedades dos
auto-ant?genos, estimula??o das outras c?lulas do sistema imunol?gico e aumento
da produ??o das citocin?nas pela ativa??o do fator nuclear de transcri??o
NF-$\kappa$B [26].

 Os componentes do sistema de complemento C3c e C4 s?o marcadores sens?veis de
um processo de inflama??o. O componente C4 participa da ativa??o da via
cl?ssica do sistema de complemento e o componente C3c tanto da via cl?ssica
como da alternativa. A queda dos nive?s do C4 e dos valores de CH50 comprova a
remiss?o do processo inflamat?rio.

 Algums resultados da nossa pesquisa diferenciam-se dos resultados obtidos
pelos outros pesquisadores. N?s n?o observamos nenhuma altera??o dos valores
nas subpopola??es dos limf?citos T CD4 e T CD8 e nem na rela??o CD4/CD8 no
nosso grupo dos pacientes com aftas, ao contr?rio, algums pesquisadores
relataram para os pacientes com aftas o aumento da percentagem dos limf?citos T
CD8, a diminui??o da subpopula??o de limf?citos T CD4 e a altera??o na rela??o
CD4/CD8 [17, 20]. E, tamb?m, ao contr?rio dos nossos resultados era observado o
aumento dos nive?s dos limf?citos B e a diminui??o dos limf?citos T HLA-DR+
[20]. Ap?s o tratamento n?s constatamos o aumento da percentagem dos limf?citos
T CD3 e B e a diminui??o da percentagem dos limf?citos T HLA-DR+ em compara??o
com os valores de refer?ncia. A queda significativa da percentagem dos
limf?citos ativados T CD3/HLA-DR+ indica a remiss?o da doen?a.

 O tratamento com Biomarine 570, que contem o ol?o de figado de tubar?o,
durante 3 meses diminuiu significativamente a frequ?ncia e a quantidade das
aftas. Infelizmente, as aftas voltaram na maioria dos pacientes ap?s a
finaliza??o do tratamento, por?m com intensidade menor. O tratamento n?o
normalizou totalmente todos os par?metros imunol?gicos, mas melhorou a resposta
dos neutr?filos para OZ e PMA,  causou o aumento da percentagem dos limf?citos
T CD3 e aproximou a percentagem dos limf?citos B e dos limf?citos T CD3/HLA-DR
aos valores de refer?ncia. Al?m disso, foi observada a diminui??o significativa
dos nive?s do componente C4 e da atividade hemol?tica CH50 do sistema de
complemento, que aproximaram-se dos valores normais ap?s tratamento.

 As Recorrentes Aftas Odontol?gicas s?o uma doen?a cr?nica e dificil de
tratar. O uso do ol?o de figado de tubar?o no tratamento dessa doen?a pode ser
justificado pela constatada melhora dos avaliados par?metros da resposta
imunol?gica e pela falta de contraindica??es e efeitos colaterais do
medicamento. O ol?o de figado do tubar?o parece ser insuficientemente eficaz
como medicamento pricipal do tratamento, mas pode ser usado como medicamento
coadjuvante com os medicamentos locais ou imunomoduladores (imunossupressores
ou imunoestimuladores) por causa da sua positiva influ?ncia no sistema
imunol?gico.

\begin{center}
\large{CONCLUS?ES}
\end{center}

 1. O tratamento com o ol?o de figado de tubar?o diminui a frequ?ncia e a
quantidade das aftas durante o tratamento e nos primeiros 2 meses ap?s
tratamento.

 2. O tratamento com o ol?o de figado de tubar?o normaliza os nive?s dos
limf?citos B e dos limf?citos T HLA-DR. Os valores do componente C4 e a
atividade hemol?tica CH50 do sistema de complemento aproximam-se aos valores
normais. A produ??o do RFT pelos neutr?filos estimulados com PMA se normaliza e
aumenta para neutr?filos estimulados com OZ em compara??o com os valores antes
de tratamento.

 3. O ol?o de figado de tubar?o pode ser administrado para pacientes com as
aftas recorrentes como medicamento coajuvante, junto com outros medicamentos
locais ou sistemicos.
\end{document}

